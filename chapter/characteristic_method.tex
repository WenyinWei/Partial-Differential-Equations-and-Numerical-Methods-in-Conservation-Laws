\part{PDE}
\section{Introduction to PDE}

This part mainly refers to \footnote{姜礼尚. 数学物理方程讲义[M]. 高等教育出版社, 1961.}.

与建立数学物理方程关系最密切的物理定律大致可以归结为两大类:1.守恒律;2.变分原理。常见的物理问题涵盖了弦振动、热传导、流体运动以及膜平衡、极小曲面等物理与几何例子。

通常我们把初始条件和边界条件统称为定解条件,再加上 pde 方程本身则可以成为一个定解问题。边界条件分为第一边界条件 (Dirichlet)、第二边界条件 (Neumann)、第三边界条件。

\subsection{Conservation Laws 守恒律}
波动方程,

\begin{equation}
\frac{\partial^2 u}{\partial t^2}-a^2\Delta u =f.
\end{equation}

热传导方程,

\begin{equation}
\frac{\partial u}{\partial t}-a^2\Delta u =f.
\end{equation}

质量守恒方程

\begin{equation}
\frac{\partial \rho}{\partial t}+\nabla\cdot (\rho \boldsymbol{v}) =f.
\end{equation}

\subsection{变分原理}

如果将函数的定义域扩大为某种集合,值域为实数域,即从一个集合到市属的映射称为泛函。例如将 $[a,b]$ 区间上的全体连续函数记为 $C([a,b])$,则对每一个该集合内的函数,都有一个对应的实数,这构成一个泛函。

通常为积分形式,

\begin{equation}
f(x) \rightarrow \int_a^b f(x)dx
\end{equation}

\begin{definition}
设 $\Omega$ 为 $bbR^2$ 中的区域,定义在 $\Omega$ 上的无穷次可微且在 $\Omega$ 的边界附近为零的函数的全体,记为 $C_0^\infty(\Omega)$.
\end{definition}

一个典型的 $bbR^2$ 上的试函数,$\rho(x,y)=ke^{-1/[1-(x^2+y^2)]}, x^2+y^2\leq 1$. 总可以选取 $k$ 使其在 $bbR^2$ 上积分为 $1$. One can expand the funciton $\rho_n(x,y)=n^2 \rho(nx,ny)$.

\begin{lemma}
Set $\Omega$ a bounded region in $bbR^2$, $f(x,y)$ continuous on $\Omega$, if for all $\phi(x,y)\in C_0^\infty (\Omega)$,

$$\int\int_{\Omega} f(x,y)\phi(x,y)dxdy=0$$,

then $f(x,y)\equiv 0$ on $\Omega$.
\end{lemma}

\subsubsection{Minimal Surface Problem}

我们有一个曲面 $S$, 它的边界被我们定死了,可以被参数化表示为一条空间闭曲线 $l$。
\begin{equation}
l:\quad\begin{aligned} x &=x(s) \\ y &=y(s) \\ u &=\varphi(s) \end{aligned} \quad\left(0 \leqslant s \leqslant s_{0}\right)
\end{equation}

\begin{equation}
J(v)=\iint_{\Omega} \sqrt{1+v_{x}^{2}+v_{y}^{2}} d x d y
\end{equation}

Now we want to get the minimal surface area funciton $u\in M_\phi=\{v|v\in C^1 (\overline{\Omega}),v|_{\partial \Omega}=\phi)\}$. 这也被称为是该变分问题的允许函数类。
\begin{equation}
J(u)=\underset{v \in M_{\phi}}{\operatorname{Min}} J(v)
\end{equation}

\subsection{定解问题的适定性}

如果一个定解问题的解 \textbf{exist, unique, stable},那么我们称这个定解问题是\textbf{适定} 的。

定解数据(如初值、边值和方程的非齐次项)一般都是通过实际测量得到的,它不可能完全正确,所以人们关心定解数据的微笑差异是否会引起解的完全失真?这是解的稳定性问题,即解是否连续依赖于定解数据?

\begin{definition}
Denote $G$ as a function set. If for each two functions $f_1,f_2\in G$, $a_1 f_1+a_2 f_2\in G (a_1,a_2\in \bbR)$, then $G$ is a linear space. If for each $f\in G$, there is a corresponding non-negative real $||f||$, and 
\begin{enumerate}
	\item if $f_1,f_2\in G$, $$||f_1+f_2||\leq ||f_1||+||f_2||$$;
	\item if $f\in G, a\in \bbR$, $$||af||=|a|~ ||f||$$;
	\item $||f||\geq 0$ holds, $||f||=0$ iff $f\equiv 0$,
\end{enumerate} 
then $G$ is 线性赋范空间, $||f||$ 称为 $f$ 的范数或模 (\textit{norm}).
\end{definition}

通常对初值的连续依赖可以表述为以下形式,

For all $\epsilon>0$, there exists $\delta>0$, when $||\phi_1-\phi_2||_\Phi<\delta$, $$||u_1-u_2||_U<\epsilon$$.



不适定的问题是客观存在的,如地质勘探、最优控制等,它们也有其研究价值。

\section{Wave Equation 波动方程}


\part{Characteristic Methods}


\section{Characteristic Methods}
\subsection{Characteristic Methods}
We have seen many conservation laws in our daily math and physcis research. Here the introduction to the PDE is omitted temporarily.
\subsubsection{Linear Equation}
\begin{example}
\begin{equation}
\left\{\begin{matrix}
	\partial_t u+a\partial_x u=0, &x\in \mathbb{R},  & t>0 \\ 
	u(x,0)=u_0(x), & x\in \mathbb{R}, & 
	\end{matrix}\right.
\end{equation}
Solution: $$x=x(t), \frac{dx}{dt}=a$$

$$u(x,t)=u(x(t),t)=z(t)$$

$$\frac{dz}{dt}=\partial_x u\cdot \frac{dx}{dt}+\partial_t u=0$$

$$
\left\{\begin{matrix}
\frac{\mathrm{d} x}{\mathrm{d} t}=a, &x(0)=x_0 ,  & x=at+x_0\\ 
\frac{\mathrm{d} z}{\mathrm{d} t}=a, & z(0)=u_0(x_0), & u(x,t)=u_0(x-at) 
\end{matrix}\right.
$$


\end{example}

Characteristic surface, $\mathcal{S}=\{(x, y, u(x, y))\}$. Its normal vector is $(\partial_x u, \partial_y u, -1)$.
\subsection{Boundary Data}
Non-characteristic boundary data is required which means that the boundary tangent direction should be \textbf{non-parallel} with the characteristic lines.

The typical boundary in 2D problem is parameterized as $\Gamma=\left\{(x, y), \cdots\},  y=\left\{\left(\gamma_{1}(s), \gamma_{2}(s)\right), r\in\cdots\right\}\right.$ The non-characteristic condition is $(a, b) \cdot\left(-\gamma_{2}^{\prime}(r), \gamma_{1}^{\prime}(r)\right) \neq 0$. $r$ serves as the position parameter showing the location of the boundary data along with characteristic line. $s$ serves as the time parameter denoting the time evolution, while $s=0$ means the initial point of the characteristic line, that is, the corresponding boundary data point.

The general linear PDE is as the following:

\begin{equation}
\left\{\begin{matrix}
a(x, y) \partial_{x} u+b(x, y) \partial_y u=c(x, y), &a, b, c \in C^{1} \\ 
\left.u(x, y)\right|_{\Gamma}=\phi(x, y), \quad &\text { when } \quad \Gamma=\left\{\left(\gamma_{1}(r), \gamma_{2}(r)\right), r \in \cdots\right\}
\end{matrix}\right.
\end{equation}

$$
\left\{\begin{array}{ll}{\frac{d x}{d s}(r,s)=a(x(s), y(s))}& x(r,0)=\gamma_1(r) \\
{\frac{d y}{d s}(r,s)=b(x(s), y(s))} & y(r,0)=\gamma_2(r) \\
{\frac{d z}{d s}(r,s)=c(x(s), y(s))} & z(r,0)=\phi(\gamma_1(r), \gamma_2(r)) \end{array}\right.
$$
\begin{lemma}
Suppose $a, b, c \in C^{1}$, $\Gamma$ is non-characteristic, 
$$
G(r, s)=\left(\begin{array}{l}{x(r, s)} \\ {y(r, s)}\end{array}\right)
$$
then
$$
\exists\ G^{-1}(x, y)=\left(\begin{array}{l}{R(x, y)} \\ {S(x, y)}\end{array}\right) \quad \text { near }(r, 0)
$$
\end{lemma}
\begin{proof}
To be filled.
\end{proof}
\subsubsection{Semilinear Equation}
A difference is made in semilinear case. $c(x,y)$ now becomes $c(x,y,u)$.

\begin{equation}
a(x, y) \partial_{x} u+b(x, y) \partial_y u=c(x, y,u)
\end{equation}

\begin{example}
$$
\left\{\begin{array}{l}{\partial_{t} u+a \partial_{x}u=u^{2}} \\ {w(x, 0)=\cos x, \quad \text{on}\quad \Gamma=\{(x, 0), x \in \mathbb{R}\}}\end{array}\right.
$$
\end{example}

\begin{example}
$$
\left\{\begin{array}{l}{\partial_x u+x \partial_y u=u} \\ {u(1, y)=h(y),\quad y \in \mathbb{R}}\end{array}\right.
$$
\end{example}

\subsubsection{Quasilinear Equation}
$a(x,y)$ and $b(x,y)$ are more complex in quasilinear case, becoming $a(x,y,u)$ and $b(x,y,u)$.

$$
\left.\left(a\left(\gamma_{1}, \gamma_{2}, \phi\right), b\left(\gamma_{1}, \gamma_{2}, \phi\right)\right)\left(-\gamma_{2}^{\prime}(r), \gamma_{1}^{\prime}( r\right)\right) \neq 0
$$

\begin{example}
$$
\left\{\begin{array}{l}{\partial_t u+u \partial_x u=0, t>0, x\in\mathbb{R}} \\ {u(x,0)=\phi(x)}\end{array}\right.
$$

Solution:

$$
\Gamma=\{(x, 0)\}=\{(r, 0), r \in \mathbb{R}\}
$$

$$
\left\{\begin{array}{ll}{\frac{d x}{d s}=u(x(r, s), t(r, s))=z(r, s),} & {x(r, 0)=r} \\ {\frac{d t}{d s}=1} & {t(r, 0)=0} \\ {\frac{d z}{d s}=0} & {z(r, 0)=\phi(r)}\end{array}\right.
$$

$$
\left\{\begin{array}{l}{x=zs+r=\phi(r) s+r} \\ {t=s} \\ {z(r, s)=\phi(r)}\end{array}\right.
$$

$$
u(x, t)=z(r,s)=\phi(x-u t)
$$
\end{example}

Readers may have noticed that the $x=zs+r=\phi(r) s+r$, which will cause problems when two characteristic lines cross each other. At this time, the method fails.
\subsubsection{Fully Nonlinear Equation}
\begin{equation}
\left\{\begin{matrix}
F(x,y,u,\partial_x u,\partial_y u)=0, &\Longrightarrow F(x,y,z,p,q)=0 \\ 
\left.u(x, y)\right|_{\Gamma}=\phi(x, y), \quad &\text { when } \quad \Gamma=\left\{\left(\gamma_{1}(r), \gamma_{2}(r)\right), r \in \cdots\right\}
\end{matrix}\right.
\end{equation}

General solution:

$$
\frac{d z}{d s}=\frac{d u}{d s}(x_r(s), y_r(s))=\partial_x u \frac{d x}{d s}+\partial_y u\frac{d y}{d s}
$$

$$
0=\frac{dF}{ds}=\left(\partial_{x} F+p\right) \frac{d x}{d s}+\left(\partial_{y} F+ q\right) \frac{d y}{d s}+\partial_{p} F \frac{d p}{d x_{1}}+\partial_{q} F \frac{d q}{d s}
$$

\subsubsection{Multi-dimensional Case}


\section{Classic Solution to Conservation Laws}
